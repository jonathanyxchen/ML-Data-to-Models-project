\documentclass[twoside,11pt]{article}

% Any additional packages needed should be included after jmlr2e.
% Note that jmlr2e.sty includes epsfig, amssymb, natbib and graphicx,
% and defines many common macros, such as 'proof' and 'example'.
%
% It also sets the bibliographystyle to plainnat; for more information on
% natbib citation styles, see the natbib documentation, a copy of which
% is archived at http://www.jmlr.org/format/natbib.pdf

%%%%%%%%%%%%%%%%%%%%%%%%%%%%%%%%%%%%%%%%%%%%%%%%
% Style file settings DO NOT MODIFY THIS SECTION
%%%%%%%%%%%%%%%%%%%%%%%%%%%%%%%%%%%%%%%%%%%%%%%%
\usepackage{jmlr2e}
\setlength{\parskip}{0.25em}
\renewcommand{\arraystretch}{1.2}

%%%%%%%%%%%%%%%%%%%%%%%%%%%%%%%%%%%%%%%%%%%%%%%%
% Handy packages
%%%%%%%%%%%%%%%%%%%%%%%%%%%%%%%%%%%%%%%%%%%%%%%%
\usepackage[normalem]{ulem}
\usepackage{amsmath}
\usepackage{amssymb, mathtools, amsfonts}
\usepackage{tikz}
\usepackage{tikz-qtree}
\usetikzlibrary{trees}
\usepackage{algorithm}
\usepackage[noend]{algpseudocode}
\usetikzlibrary{automata,positioning}
\usepackage{multirow}
\allowdisplaybreaks

%%%%%%%%%%%%%%%%%%%%%%%%%%%%%%%%%%%%%%%%%%%%%%%%
% Definitions of handy macros can go here
%%%%%%%%%%%%%%%%%%%%%%%%%%%%%%%%%%%%%%%%%%%%%%%%

% Operators for drawing in-text edges for PAGs, DAGs, ADMGs, CGs etc
\DeclareMathOperator{\circlearrow}{\hbox{$\circ$}\kern-1.5pt\hbox{$\rightarrow$}}
\DeclareMathOperator{\circlecircle}{\hbox{$\circ$}\kern-1.2pt\hbox{$--$}\kern-1.5pt\hbox{$\circ$}}
\DeclareMathOperator{\diedgeright}{\textcolor{blue}{\boldsymbol{\rightarrow}}}
\DeclareMathOperator{\diedgeleft}{\textcolor{blue}{\boldsymbol{\leftarrow}}}
\DeclareMathOperator{\biedge}{\textcolor{red}{\boldsymbol{\leftrightarrow}}}
\DeclareMathOperator{\udedge}{\textcolor{brown}{\boldsymbol{\textendash}}}

% Operators for relations in graphs
\DeclareMathOperator{\an}{an}
\DeclareMathOperator{\pa}{pa}
\DeclareMathOperator{\ch}{ch}
\DeclareMathOperator{\pre}{pre}
\DeclareMathOperator{\de}{de}
\DeclareMathOperator{\nd}{nd}
\DeclareMathOperator{\sib}{sib}
\DeclareMathOperator{\dis}{dis}
\DeclareMathOperator{\mb}{mb}

% Operator for ``do''
\DeclareMathOperator{\doo}{do}

% Operator for odds ratio
\DeclareMathOperator{\odds}{\text{OR}}

% Operators for optimization problems
\DeclareMathOperator*{\argmax}{arg\,max}
\DeclareMathOperator*{\argmin}{arg\,min}

% Operators for independence, expectation and calligraphy G
\def\ci{\perp\!\!\!\perp}
\newcommand{\E}{\mathbb{E}}
\newcommand{\G}{\mathcal{G}}


%%%%%%%%%%%%%%%%%%%%%%%%%%%%%%%%%%%%%%%%%%%%%%%%%%%%%%%%%%%%%%%%%%
% Pick and abbreviated title for your project and put down your
% last name. This title and your name will appear alternating
% pages of the project writeup at the top
%%%%%%%%%%%%%%%%%%%%%%%%%%%%%%%%%%%%%%%%%%%%%%%%%%%%%%%%%%%%%%%%%%
\ShortHeadings{An Abbreviated Title For Your Project If Needed}{Put your last name here}
\firstpageno{1}

\begin{document}

%%%%%%%%%%%%%%%%%%%%%%%%%%%%%%%%%%%%%%%%%%%%%%%%%%%%%%%%%%%%%%%%%%
% Insert the title of your project and details abouy you here
%%%%%%%%%%%%%%%%%%%%%%%%%%%%%%%%%%%%%%%%%%%%%%%%%%%%%%%%%%%%%%%%%%
\title{Title of Your Project
}

\author{\name {Put your name here}
		\email {email here} \\
       	\addr {Department/major here} \\
      	{jhed here} 
	}


\maketitle

%%%%%%%%%%%%%%%%%%%%%%%%
% Introduction
%%%%%%%%%%%%%%%%%%%%%%%%
\section{Introduction}
\label{sec:intro}
Introduce high-level background and highlights of your analysis here$\dots$ Example of citing your data sources. I used protein/phospholipid expression data from \cite{sachs2005causal}. Citations without parantheses when it flows naturally in the text use the \texttt{$\backslash$cite} command. Citations with the parantheses use the \texttt{$\backslash$citep} command.


%%%%%%%%%%%%%%%%%%%%%%%%
% Preliminaries
%%%%%%%%%%%%%%%%%%%%%%%%
\section{Preliminaries}
\label{sec:prelims}
Technical background for explaining your project goes here$\dots$ 

A compressed example of how to provide technical background for causal models of a directed acyclic graph (DAG) is given below.

Causal models of a DAG $\G$ defined over a set of variables $V$ may be interpreted as a tuple consisting of the DAG itself, and a system of non-parametric structural equations with independent errors equipped with the do-operator. Each variable is determined as a function of its parents and an independent error term. This induces a distribution $p(V)$ that factorizes according to the DAG $\G$ as follows,
%
\begin{align*}
p(V) = \prod_{V_i \in V} p(V_i \mid \pa_\G(V_i)),
\end{align*}
%
where $\pa_\G(V_i)$ denotes the parents of $V_i$ in $\G.$ Under this interpretation, a directed edge $V_i \diedgeright V_j$ may be interepreted as saying that $V_i$ is potentially a direct cause of $V_j$. Conditional independences in $p(V)$ can be read off from the DAG via d-separation, i.e., $(X \ci Y \mid Z)_\text{d-sep} \implies (X \ci Y \mid Z)_{\text{in } p(V)}.$ To facilitate structure learning, I will restrict my analysis to the set of \emph{faithful} distributions where $(X \ci Y \mid Z)_\text{d-sep} \iff (X \ci Y \mid Z)_{\text{in } p(V)}.$ In addition, I make the simplifying assumption that the relations between my variables are linear.

%%%%%%%%%%%%%%%%%%%%%%%%
% Methods
%%%%%%%%%%%%%%%%%%%%%%%%
\section{Methods}
\label{sec:methods}
Introduce specifics of your analysis and methods you used here$\dots$ The following sentences provide contrived examples of how to  state and cite a method you used in your analysis. 

Greedy Equivalence Search (GES), a score-based method for learning DAGs \citep{chickering2002optimal}, was used to learn an equivalence class of possible causal structures with no unmeasured confounders.

A non-parametric conditional independence test known as the Fast Conditional Independence Test was used to check for conditional independences \citep{chalupka2018fast}.


%%%%%%%%%%%%%%%%%%%%%%%%
% Results
%%%%%%%%%%%%%%%%%%%%%%%%
\section{Results}
\label{sec:results}
Write up and visualize your results here $\dots$

%%%%%%%%%%%%%%%%%%%%%%%%%
% Discussion + Conclusion
%%%%%%%%%%%%%%%%%%%%%%%%%
\section{Discussion and Conclusion}
\label{sec:discussion}
Provide a discussion on the results of your analysis based on results presented in Section~\ref{sec:results}$\dots$


%%%%%%%%%%%%%%%%%%%%%%%%%
% References
%%%%%%%%%%%%%%%%%%%%%%%%%
\newpage
\bibliography{references}


\end{document}

